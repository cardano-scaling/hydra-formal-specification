\chapter{Protocol Overview}\label{sec:overview}

The Hydra Head protocol provides functionality to lock a set of UTxOs on a
blockchain, referred to as the \emph{mainchain}, and evolve it inside a
so-called off-chain \emph{head}, independently of the mainchain. At any point,
the head can be closed with the effect that the locked set of UTxOs on the
mainchain is replaced by the latest set of UTxOs inside the head. The protocol
guarantees full wealth preservation: no generation of funds can happen off-chain
(inside a head) and no responsive honest party involved in a head can ever lose
any funds other than by consenting to give them away. In exchange for decreased
liveness guarantees (stop any time), it can essentially proceed at network speed
under good conditions, thereby reducing latency and increasing throughput. At the
same time, the head protocol provides the same capabilities as the mainchain by
reusing the same ledger model and transaction formats --- making the protocol
"isomorphic".

\subsection{Opening the head}

To create a head-protocol instance, any party may take the role of an
\emph{initiator} and ask other parties, the \emph{head members}, to participate
in the head by exchanging public keys and agreeing on other protocol parameters.
These public keys are used for both, the authentication of head-related on-chain
transactions that are restricted to head members (e.g., a non-member is not
allowed to close the head) and for signing off-chain transactions in the head.

The initiator then establishes the head by submitting an \emph{initial}
transaction to the mainchain that contains the Hydra protocol parameters and mints special
\emph{participation tokens (PT)} identifying the head members. The
\emph{initial} transaction also initializes a state machine (see
Fig.~\ref{fig:SM_states_basic}) that manages the ``transfer'' of UTxOs into the
head and back. The state machine comprises the four states: $\stInitial$,
$\stOpen$, $\stClosed$, and $\stFinal$. A \emph{state thread token (ST)} minted
in \emph{initial} marks the head output and ensures contract continuity~\cite{eutxo}.

\begin{figure}[t!]
  \centering
  \begin{tikzpicture}[>=stealth,auto,node distance=2.8cm, initial text=$\mathsf{init}$, every
    state/.style={text width=10mm, text height=2mm, align=center}]
    \node[state, initial] (initial) {$\stInitial$};
    \node[state] (open) [above right of=initial] {$\stOpen$};
    \node[state] (closed) [right of=open] {$\stClosed$};
    \node[state] (final) [below right of=closed] {$\stFinal$};

    \path[->] (initial) edge [bend left=20] node {$\stCollect$} (open);
    \path[->] (open) edge [bend left=20] node {$\stClose$} (closed);
    \path[->] (open) edge [loop above] node {$\mathsf{increment}$} (open);
    \path[->] (open) edge [loop below] node {$\mathsf{decrement}$} (open);
    \path[->] (closed) edge [bend left=20] node {$\stFanout$} (final);
    \path[->] (closed) edge [loop above] node {$\stContest$} (closed);
    \path[->] (initial) edge [bend right=20] node {$\stAbort$} (final);
  \end{tikzpicture}

  \caption{Mainchain state diagram for this version of the Hydra protocol.}\label{fig:SM_states_basic}
\end{figure}

%%% Local Variables:
%%% mode: latex
%%% TeX-master: "main"
%%% End:


Once the initial transaction appears on the mainchain, establishing the initial
state $\stInitial$, each head member can attach a \mtxCom{} transaction, which
locks (on the mainchain) the UTxOs that the party wants to commit to the head,
or deliberately acknowledges to commit nothing.

The commit transactions are subsequently collected by the \mtxCCom{} transaction
causing a transition from $\stInitial$ to $\stOpen$. Once the $\stOpen$ state is
confirmed, the head members start running the off-chain head protocol, which
evolves the initial UTxO set (the union over all UTxOs committed by all head
members) independently of the mainchain. For the case where some head members
fail to post a \mtxCom{} transaction, the head can be aborted by going directly
from $\stInitial$ to $\stFinal$ \todo{increment / decrement}.


\subsection{The Coordinated Head protocol}

The actual Head protocol starts after the initialization phase with an initial
set of UTxOs that is identical to the UTxOs locked on-chain via the \mtxCom{}
and \mtxCCom{} transactions.

The protocol processes off-chain transactions by distributing them between participants,
while each party maintains their view of the local UTxO state. That is, the current
set of UTxOs evolved from the initial UTxO set by applying transactions as they
are received from the other parties.

To confirm transactions and allow for an on-chain decommit of the resulting UTxO
set without needing the whole transaction history, snapshots are created by the
protocol participants. The initial snapshot $U_{0}$ corresponds to the initial
UTxO set, while snapshots thereafter $\Uset_1,\Uset_2,\ldots$ are created with
monotonically increasing snapshot numbers.

For this, the next snapshot leader (round-robin) requests his view of a new confirmed state to be
signed by all participants as a new snapshot. The leader does not need to send his local state,
but only indicate, by hashes, the set of transactions to be included in order to
obtain the to-be-snapshotted UTxO set.

The other participants sign the snapshot as soon as they have (also) seen the
transactions that are to be processed on top of its preceding snapshot: a
party's local state is always ahead of the latest confirmed snapshot.

Signatures are broadcast and aggregated by each party. When all signature parts
of the multi-signature are received and verified, a snapshot is considered
confirmed. As a consequence, a participant can safely delete (if wished) all
transactions that have been processed into it as the snapshot's multi-signature
is now evidence that this state once existed during the head evolution.

\subsection{Closing the head}

The head protocol is designed to allow any head member at any point in time to
produce, without interaction, a certificate to close the head. This certificate
is created from the latest confirmed snapshot, specifically from its snapshot
number and the respective multisignature. Using this certificate, the head
member may ``force close'' the head by advancing the mainchain state machine to
the $\stClosed$ state.

Once in $\stClosed$, the state machine grants parties a contestation period,
during which parties may contest the closure by posting the certificate of a
newer snapshot on-chain in a contest transaction. Contesting leads back to the state
$\stClosed$ and each party can contest at most once. After the contestation period has elapsed, the state machine may
proceed to the $\stFinal$ state. The state machine enforces that the outputs of
the transaction leading to $\stFinal$ correspond exactly to the latest UTxO set
seen during the contestation period.

\subsection{Differences}
% TODO More details on what was in orig paper?
In the Coordinated Head protocol, off-chain consensus is simplified by not
having transactions confirmed concurrently to the snapshots (and to each other)
but having the snapshot leader propose, in their snapshot, a set of transactions
for explicit confirmation. The parties' views of confirmed transactions thus
progress in sync with each other (once per confirmed snapshot), thus simplifying
the close/contest procedure on the mainchain. Also, there is no need for
conflict resolution as in Appendix~B of~\cite{hydrahead20}. In summary, the
differences to the original Head protocol in~\cite{hydrahead20} are:

\begin{itemize}
	\item No hanging transactions due to `coordination'.
	\item No acknowledgement nor confirmation of transactions.
	\item No confirmation message for snapshots (two-round local confirmation).
    \todo{increment / decrement extension}
\end{itemize}

%%% Local Variables:
%%% mode: latex
%%% TeX-master: "main"
%%% End:
