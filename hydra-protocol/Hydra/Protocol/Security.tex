\section{Security (WIP --- Iteration 1)}\label{sec:security}
\todo{The security analysis is still \textbf{sketchy}, with the goal to make it more formal in upcoming iterations}

\todo{Add security experiment}
\noindent Adversaries:

\begin{mdescription}
\item[Active Adversary.] An \emph{active adversary} $\adv$ has full control
  over the protocol, i.e., he is fully unrestricted in the above\todo{above this section there is no security game} security game.

 \item[Network Adversary.] A \emph{network adversary} $\adv_\emptyset$ does not corrupt
   any head parties, eventually delivers all sent network messages
   (i.e., does not drop any messages), and does not cause the $\hpClose$ event.
   Apart from this restriction, the adversary can act arbitrarily in the above experiment.
\end{mdescription}

\noindent Random variables:

\begin{mitemize}
\item $\That_i$: the set of transactions $\tx$ for which party $\party_i$,
  \emph{while uncorrupted}, output $(\hpSeen,\tx)$;

\item $\Tbar_i$: the set of transactions $\tx$ for which party $\party_i$,
  \emph{while uncorrupted}, output $(\hpConf,\tx)$;
    
\item $\Snapbar_i$: latest snapshot $(s,U)$ that party
  $\party_i$ performed \emph{while uncorrupted}: output $(\hpSnap,(s,U))$;

\item $\Hcont$: the set of (at the time) uncorrupted parties who produced
  $\xi$ upon close/contest request and $\xi$ was applied to
  correct~$\eta$; and

\item $\honest$: the set of parties that remain uncorrupted.
\end{mitemize}


\noindent Security conditions / events:

\begin{itemize}
\item \propName{Consistency (Head)}: In presence of an active adversary, the
  following condition holds at any point in time:
   For all $i,j$,
   $\Uinit \circ (\Tbar_i \cup \Tbar_j) \not= \bot$, i.e., no two
   uncorrupted parties see conflicting transactions confirmed.

  \item \propName{Oblivious Liveness (Head)}:
    Consider any protocol execution in presence of a network adversary wherein
    the head does not get closed for a sufficiently long period of time, and consider
    an honest party $p_i$ who enters transaction $\tx$ by executing $(\hpNew,\tx)$ \emph{each time after having finished a snapshot}.

    Then the following eventually holds:
    $\tx \in \bigcap_{i\in[n]} \Tbar_i\ \vee\ %
    \forall i: \Uinit \circ (\Tbar_i\cup\{\tx\}) = \bot$,
    i.e., every party will observe the transaction confirmed or every party
    will observe the transaction in conflict with their confirmed transactions.\footnote{
      In particular, \emph{liveness} expresses that the protocol makes progress
      under reasonable network conditions if no head parties get corrupted.
    }

\item \propName{Soundness (Chain)}: In presence of an active adversary,
  the following condition is satisfied:
  $\exists \Ttilde \subseteq \bigcap_{i \in \honest} \That_i : \Ufinal
  = \Uinit \circ \Ttilde\not=\bot$, i.e., the final UTxO set results
  from applying a set of transactions to $U_0$ that have been seen by
  all honest parties (wheras each such transaction applies conforming to the ledger rules).
\item \propName{Completeness (Chain)}: In presence of an active adversary,
  the following condition holds: For $\Ttilde$ as above,
  $\bigcup_{p_i \in \Hcont} \Tbar_i \subseteq \Ttilde$, i.e., all
  transactions seen as confirmed by an honest party at the end of the
  protocol are considered.
\end{itemize}

Note that the original version of the coordinated head satisfies a stronger version of liveness which is important for the 'user experience' in the protocol:
\begin{itemize}
  \item \propName{Liveness (Head)}:
    Consider any protocol execution in presence of a network adversary wherein
    the head does not get closed for a sufficiently long period of time, and consider
    an honest party $p_i$ who enters transaction $\tx$ by executing $(\hpNew,\tx)$.

    Then the following eventually holds:
    $\tx \in \bigcap_{i\in[n]} \Tbar_i\ \vee\ %
    \forall i: \Uinit \circ (\Tbar_i\cup\{\tx\}) = \bot$,
    i.e., every party will observe the transaction confirmed or every party
    will observe the transaction in conflict with their confirmed transactions.\footnote{
      In particular, \emph{liveness} expresses that the protocol makes progress
      under reasonable network conditions if no head parties get corrupted.
    }
\end{itemize}


\subsection{Proofs}

\paragraph{Consistency.}

\begin{lemma}[Consistency]
  \label{lem:consistency}
  The coordinated head protocol satisfies the \propName{Consistency} property.
\end{lemma}
\begin{proof}
  Observe that $\Tbar_i\cup\Tbar_j\subseteq\That_i$ since no
  transaction can be confirmed without every honest party signing off
  on it.  Since parties do not sign conflicting transactions
  (see $\hpRS$, `wait'), we have
  $\Uinit\applytx\Tbar_i\neq\bot$,
  $\Uinit\applytx\Tbar_j\neq\bot$, and
  $\Uinit\applytx\That_i\neq\bot$.  Thus, since $\Tbar_i\cup\Tbar_j\subseteq\That_i$
  it follows that
  $\Uinit\applytx(\Tbar_i\cup\Tbar_j)\neq\bot$
\end{proof}

\paragraph{Oblivious Liveness.}
For all lemmas towards oblivious liveness, we assume the presence of a network adversary, and that the head does not get closed for a sufficiently
long period of time.
We call this the \emph{liveness condition}.

\begin{lemma}\label{lem:reqconf}  
  Under the liveness condition, any snapshot issued as $(\hpRS,s,T)$ will eventually be confirmed
  in the sense that every party holds a valid mulisignature on it.
\end{lemma}
\begin{proof}
  Consider a party $p_i$ receiving message $(\hpRS,s,T)$. We demonstrate that $p_i$ executes
  the code past the `wait' instruction of the $\hpRS$ routine. 

  \begin{itemize}
   \item Passing the `require' guard:
  Note that the snapshot leader sends the request only if $\hats=\bars$, and for $s=\hats+1$.
  Thus, $\hats_i=\hats$ since $p_i$ has already signed the snapshot for $\hats$. The `require'
  guard is thus satisfied for $p_i$.

   \item Passing the `wait' guard:
  Since the snapshot leader sees $\hats=\bars$, also $p_i$ will eventually see $\hats_i=\bars_i$. Furthermore, since all leaders are honest, it holds that $\hatmU\applytx\mT_{res}\not=\bot$ by construction.
  \end{itemize}

  This implies that every party will eventually sign and acknowledge the newly created snapshot.
  Finally, the `require' and `wait' guards of the $\hpAS$ code will be passed by every party
  since an $\hpAS$ for snapshot number $s$ can only be received for $s\in\{\hats,\hats+1\}$
  as an acknowledgement can only be received for the current snapshot being worked on by $p_i$
  or a snapshot that is one step ahead---implying that everybody will hold a valid multisignature
  on the snapshot in consideration.
\end{proof}

\begin{lemma}[Eternal snapshot confirmation]\label{lem:eternal}
  Under the liveness condition, as long as new transactions are issued, for any $k>0$, every party eventually confirms
  a snapshot with sequence number $s=k$.
\end{lemma}
\begin{proof}
  By Lemma~\ref{lem:reqconf}, any requested snapshot eventually gets confirmed, implying
  that the next leader observes $\hats=\bars$ and thus, in turn, issues a new snapshot.
  Thus, for any $k$, a snapshot is eventually confirmed.
\end{proof}

\begin{lemma}[Oblivious Liveness]
  \label{lem:liveness}
  The coordinated head protocol satisfies the \propName{Oblivious Liveness} property.
\end{lemma}
\begin{proof}
  Consider the first point in time where a transaction $\tx$ enters the system by some party $p_i$
  issuing $(\hpNew,\tx)$, and consider the next point in time
  $t$ when $p_i$ issues a snapshot.

  By Lemma~\ref{lem:eternal}, this snapshot will eventually be issued and confirmed by all parties.
  
  \medskip
  
  Let $\hatmT$ be the transactions to be considered by $p_i$'s snapshot: $\hatmL=\barmU\applytx\hatmT$
  where $\barmU$ is the snapshot prior to $p_i$'s. Since $p_i$ issues
  $(\hpRT,\tx)$ after each snapshot, we have that, either,
  \begin{itemize}
      \item $\tx\in\hatmT$, in which case $\tx \in \bigcap_{i\in[n]} \Tbar_i$ after everybody has completed this snapshot, or,
      \item $\tx\notin\hatmT$, in which case $\hatmL\applytx\tx=\bot$ ($\tx$ is still in the wait queue of $(\hpRT,\tx)$. After everybody has completed this snapshot, it thus holds that $\forall i: \Uinit\applytx\Tbar_i=\hatmL$, and thus, that
      $\forall i: \Uinit\applytx(\Tbar_i\cup\{\tx\})=\bot$.
  \end{itemize}
  In both cases, the lemma follows.
\end{proof}

\paragraph{Soundness and completeness.}

\begin{lemma}[Soundness]
  \label{lem:soundness}
  The basic head protocol satisfies the \propName{Soundness} property.
\end{lemma}

\begin{proof}
  Let $T$ be the set of transactions such that $\Ufinal=U_0\applytx T$.
  Since $\Ufinal$ is multi-signed, it holds that $T\subseteq\That_i$
  ($T$ is \emph{seen}) by every honest party in the head.
  Furthermore, since honest signatures are only issued for valid transaction,
  $\Ufinal\not=\bot$ (i.e., $\Ufinal$ is a valid state), and soundness
  follows.
\end{proof}


\begin{lemma}[Completeness]
 \label{lem:completeness}
 The basic head protocol satisfies the \propName{Completeness}
 property.
\end{lemma}
\begin{proof}
  Consider all parties $p_i\in\Hcont$. Since the close/contest process
  finally accepts the latest multi-signed snapshot, it holds that
  $\Ufinal.s \geq \max_{p_i\in\Hcont}(\bars_i)$, and thus that
  $\bigcup_{p_i\in\Hcont}\Tbar_i\subseteq\bigcap_{p_i\in\honest}\That_i$,
  and completeness follows.
\end{proof}

%%% Local Variables:
%%% mode: latex
%%% TeX-master: "main"
%%% End:
