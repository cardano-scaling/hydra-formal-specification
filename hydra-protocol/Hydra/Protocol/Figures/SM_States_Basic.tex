\begin{figure}[t!]
  \centering
  \begin{tikzpicture}[>=stealth,auto,node distance=2.8cm, initial text=$\mathsf{init}$, every
    state/.style={text width=10mm, text height=2mm, align=center}]
    \node[state, initial] (initial) {$\stInitial$};
    \node[state] (open) [above right of=initial] {$\stOpen$};
    \node[state] (closed) [right of=open] {$\stClosed$};
    \node[state] (final) [below right of=closed] {$\stFinal$};

    \path[->] (initial) edge [bend left=20] node {$\stCollect$} (open);
    \path[->] (open) edge [bend left=20] node {$\stClose$} (closed);
    \path[->] (open) edge [loop above] node {$\mathsf{increment}$} (open);
    \path[->] (open) edge [loop below] node {$\mathsf{decrement}$} (open);
    \path[->] (closed) edge [bend left=20] node {$\stFanout$} (final);
    \path[->] (closed) edge [loop above] node {$\stContest$} (closed);
    \path[->] (initial) edge [bend right=20] node {$\stAbort$} (final);
  \end{tikzpicture}

  \caption{Mainchain state diagram for this version of the Hydra protocol.}\label{fig:SM_states_basic}
\end{figure}

%%% Local Variables:
%%% mode: latex
%%% TeX-master: "main"
%%% End:
