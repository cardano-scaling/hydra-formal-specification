\clearpage
\section{On-chain Protocol}\label{sec:on-chain}
\todo{Update figures}

\todo{Open problem: ensure abort is always possible. e.g. by individual aborts or undoing commits}
\todo{Open problem: ensure fanout is always possible, e.g. by limiting complexity of $U_0$}

\noindent The following sections describe the the \emph{on-chain} protocol
controlling the life-cycle of a Hydra head, which can be intuitively described
as a state machine (see Figure~\ref{fig:head-protocol-states}). Each transition
in this state machine is represented and caused by a corresponding Hydra
protocol transaction
on-chain: \mtxInit{}~\ref{sec:init-tx}, \mtxCom{}~\ref{sec:commit-tx}, \mtxAbort{}~\ref{sec:abort-tx}, \mtxCollect{}~\ref{sec:collect-tx}, \mtxIncrement{}~\ref{sec:increment-tx}, \mtxDecrement{}~\ref{sec:decrement-tx}, \mtxClose{}~\ref{sec:close-tx}, \mtxContest{}~\ref{sec:contest-tx}, and \mtxFanout{}~\ref{sec:fanout-tx}. \\

\noindent Besides the main state transitions of the head protocol, there is
  the related ``deposit protocol'' with two transactions in support of
  \mtxIncrement{}: \mtxDeposit{}~\ref{sec:deposit-tx} and \mtxRecover{}~\ref{sec:recover-tx}. \\

% TODO: Could include a combined overview, slightly more detailed than Figure 1
% of the transaction trace for the full life cycle maybe?

\noindent The head protocol defines one minting policy script and three
validator scripts:
\begin{itemize}
  \item $\muHead$ governs minting of state and participation tokens in
  $\mtxInit{}$ and burning of these tokens in $\mtxAbort{}$ and
  $\mtxFanout{}$.
  \item $\nuInitial$ controls how UTxOs are committed to the head in
  $\mtxCommit{}$ or when the head initialization is aborted via
  $\mtxAbort{}$.
  \item $\nuCommit$ controls the collection of committed UTxOs into the head in
  $\mtxCollect$ or that funds are reimbursed in an $\mtxAbort{}$.
  \item $\nuHead$ represents the main protocol state machine logic and ensures
  contract continuity throughout $\mtxCollect{}$, $\mtxDecrement{}$,
  \mtxIncrement{}, $\mtxClose{}$, $\mtxContest{}$ and
  $\mtxFanout{}$.
\end{itemize}

\noindent The deposit protocol defines one validator script:
  \begin{itemize}
	\item $\nuDeposit$ controls that \mtxDeposit{} transaction output is
	claimed correctly into a head via \mtxIncrement{} or recovered after
	the deadline has passed in a \mtxRecover{} transaction.
  \end{itemize}

\subsection{Init transaction}\label{sec:init-tx}

The \mtxInit{} transaction creates a head instance and establishes the initial
state of the protocol and is shown in Figure~\ref{fig:initTx}. The head
instance is represented by the unique currency identifier $\cid$ created by
minting tokens using the $\muHead$ minting policy script which is parameterized
by a single output reference parameter $\seed \in \tyOutRef$:
\[
  \cid = \hash(\muHead(\seed))
\]

\begin{figure}
  \centering
  \includesvg[width=0.8\textwidth]{Hydra/Protocol/Figures/initTx.svg}
  \caption{\mtxInit{} transaction spending a seed UTxO, and producing the head
	output in state $\stInitial$ and initial outputs for each participant.}\label{fig:initTx}
\end{figure}

\noindent Two kinds of tokens are minted:
\begin{itemize}
  \item A single \emph{State Thread (ST)} token marking the head output. This
  output contains the state of the protocol on-chain and the token ensures
  contract continuity. The token name is the well known string
  \texttt{HydraHeadV1}, i.e.
  $\st = \{\cid \mapsto \texttt{HydraHeadV1} \mapsto 1\}$.
  \item One \emph{Participation Token (PT)} per participant
  $i \in \{1 \dots |\hydraKeys|  \}$ to be used for authenticating further
  transactions and to ensure every participant can commit and cannot be
  censored. The token name is the participant's verification key hash
  $\keyHash_{i} = \hash(\msVK_{i})$ of the verification key as received
  during protocol setup, i.e.
  $\pt_{i} = \{\cid \mapsto \keyHash_{i} \mapsto 1\}$.
\end{itemize}

\noindent Consequently, the \mtxInit{} transaction
\begin{itemize}
  \item has at least input $\seed$,
  \item mints the state thread token $\st$, and one $\pt$ for each of the $|\hydraKeys|$
  participants with policy $\cid$,
  \item has $|\hydraKeys|$ initial outputs $o_{\mathsf{initial}_{i}}$ with datum $\datumInitial{} = \cid$,
  \item has one head output
  $o_{\mathsf{head}}$, which captures
  the initial state of the protocol in the datum
  \[
	\datumHead = (\stInitial,\cid',\seed',\hydraKeys,\Tcontest)
  \]
  where
  \begin{mitemize}
	\item $\stInitial$ is a state identifier,
	\item $\cid'$ is the unique currency id of this instance,
	\item $\seed'$ is the output reference parameter of $\muHead$,
	\item $\hydraKeys$ are the off-chain multi-signature keys from the setup
	phase,
	\item $\Tcontest$ is the contestation period.
  \end{mitemize}
\end{itemize}

\noindent The $\muHead(\seed)$ minting policy is the only script that verifies
init transactions and can be redeemed with either a $\mathsf{Mint}$ or
$\mathsf{Burn}$ redeemer:
\begin{itemize}
  \item When evaluated with the $\mathsf{Mint}$ redeemer,
  \begin{menumerate}
	\item The seed output is spent in this transaction. This guarantees uniqueness of the policy $\cid$ because the EUTxO ledger ensures that $\seed$ cannot be spent twice in the same chain.
	$(\seed, \cdot) \in \txInputs$
	\item All entries of $\txMint$ are of this policy and of single quantity $\forall \{c \mapsto \cdot \mapsto q\} \in \txMint : c = \cid \land q = 1$
	\item Right number of tokens are minted $|\txMint| = |\hydraKeys| + 1$
	% TODO: |\txMint| may not be clear to the reader, maybe combine with item above, but be more explicit.
	\item State token is sent to the head validator $\st \in \valHead$ % TODO: fact that it is goverend by nuHead is a bit implicit here.
	\item \textcolor{blue}{The correct number of initial outputs are present $|(\cdot, \nuInitial, \cdot) \in \txOutputs| = |\hydraKeys|$}
	% XXX: this is implied by the ledger, so can be removed
	\item All participation tokens are sent to the initial validator as an initial output $\forall i \in [1 \dots |\hydraKeys|] : \{\cid \mapsto \cdot \mapsto 1\} \in \valInitial{i}$
	\item The $\datum_{\mathsf{head}}$ contains own currency id $\cid = \cid'$ and the right seed reference $\seed = \seed'$
	\item All initial outputs have a $\cid$ as their datum: $\forall i \in [1 \dots |\hydraKeys|] : \cid = \datumInitial{i}$
  \end{menumerate}
  \item When evaluated with the $\mathsf{Burn}$ redeemer,\todo{move to abort/fanout?}
  \begin{menumerate}
	\item All tokens for this policy in $\txMint$ need to be of negative quantity
	$\forall \{\cid \mapsto \cdot \mapsto q\} \in \txMint : q < 0$.
  \end{menumerate}
\end{itemize}

\noindent \textbf{Important:} The $\muHead$ minting policy only ensures
uniqueness of $\cid$, that the right amount of tokens have been minted and sent
to $\nuHead$ and $\nuInitial$ respectively, while these validators in turn
ensure continuity of the contract. However, it is \textbf{crucial} that all head
members check that head output always contains an $\st$ token of policy $\cid$
which satisfies $\cid = \hash(\muHead(\seed))$. The $\seed$ from a head datum
can be used to determine this. Also, head members should verify whether the
correct verification key hashes are used in the $\pt$s and the initial state is
consistent with parameters agreed during setup. See the initialTx behavior in
Figure~\ref{fig:off-chain-prot} for details about these checks.\\

\subsection{Commit Transaction}\label{sec:commit-tx}

A \mtxCom{} transaction may be submitted by each participant
$\forall i \in \{1 \dots |\hydraKeys|\}$ to commit some UTxO into the head or
acknowledge to not commit anything. The transaction is depicted in
Figure~\ref{fig:commitTx} and has the following structure:
\begin{itemize}
  \item One input spending from $\nuInitial$ with datum $\datumInitial{}$,
  where value $\valInitial{i}$ holds a $\pt_i$, and the redeemer
  $\redeemerInitial{} \in \tyOutRef^{*}$ is a list of output
  references to be committed,
  \item zero or more inputs with reference $\txOutRef_{\mathsf{committed}_{j}}$
  spending output $o_{\mathsf{committed}_{j}}$ with
  $\val_{\mathsf{committed}_{j}}$,
  \item one output paying to $\nuCommit$ with value $\valCommit{i}$ and datum $\datumCommit{}$.
\end{itemize}

\noindent The $\nuInitial$ validator with $\datumInitial{} = \cid$ and
$\redeemerInitial{} = \underline{\txOutRef}_{\mathsf{committed}}$ ensures that:
\begin{menumerate}
  \item All committed value is in the output
  $\valCommit{i} \supseteq \valInitial{i} \cup (\bigcup_{j=1}^{m} \val_{\mathsf{committed}_{j}})$
  \footnote{The $\supseteq$ is important for real world situations where the values
	might not be exactly equal due to ledger constraints (i.e. to ensure a
	minimum value on outputs).}
  \item Currency id and committed outputs are recorded in the output datum
  $\datumCommit{} = (\cid, C_{i})$, where
  $C_{i} = \forall j \in \{1 \dots m\} : [(\txOutRef_{\mathsf{committed}_{j}},\bytes(o_{\mathsf{committed}_{j}}))]$
  is a list of all committed UTxO recorded as tuples on-chain.
  \item Transaction is signed by the right participant
  $\exists \{\cid \mapsto \keyHash_{i} \mapsto 1\} \in \valInitial{} \Rightarrow \keyHash_{i} \in \txKeys$
  \item No minting or burning $\txMint = \varnothing$
\end{menumerate}

\noindent The $\nuCommit$ validator ensures the output is collected by either a
\mtxCCom{} in Section~\ref{sec:collect-tx} or \mtxAbort{} in
Section~\ref{sec:abort-tx} transaction of the on-chain state machine, selected
by the appropriate redeemer.

\begin{figure}
  \centering
  \includesvg[width=0.8\textwidth]{Hydra/Protocol/Figures/commitTx.svg}
  \caption{\mtxCom{} transaction spending an initial output and a single
	committed output, and producing a commit output.}\label{fig:commitTx}
\end{figure}
\todo{update with multiple commits}

\subsection{Abort Transaction}\label{sec:abort-tx}

The \mtxAbort{} transaction (see Figure~\ref{fig:abortTx}) allows a
party to abort the creation of a head and consists of
\begin{itemize}
  \item one input spending from $\nuHead$ holding the $\st$ with $\datumHead$,
  \item $\forall i \in \{1 \dots |\hydraKeys|\}$ inputs either
  \begin{itemize}
	\item spending from $\nuInitial$ with with $\pt_{i} \in \valInitial{i}$ and $\datumInitial{i} = \cid$, or
	\item spending from $\nuCommit$ with with $\pt_{i} \in \valCommit{i}$ and $\datumCommit{i} = (\cid, C_{i})$,
  \end{itemize}
  \item outputs $o_{1} \dots o_{m}$ to redistribute already committed UTxOs.
\end{itemize}

Note that \mtxAbort{} represents a final transition of the state
machine and hence there is no state machine output.

\noindent Each spent $\nuInitial$ validator with $\datumInitial{i} = \cid$ and $\redeemerInitial{i} = \mathsf{abort}$ ensures that:
\begin{menumerate}
  \item The state token of currency $\cid$ is getting burned $\{\st \mapsto -1\} \subseteq \txMint$.
\end{menumerate}

\noindent Each spent $\nuCommit$ validator with $\datumCommit{i} = (\cid,\cdot)$ and $\redeemerCommit{i} = \mathsf{abort}$ ensures that:
\begin{menumerate}
  \item The state token of currency $\cid$ is getting burned $\{\st \mapsto -1\} \subseteq \txMint$.
\end{menumerate}

\noindent The $\muHead(\seed)$ minting policy governs the burning of tokens via
redeemer $\mathsf{burn}$ that:
\begin{menumerate}
  \item All tokens in $\txMint$ need to be of negative quantity
  $\forall \{\cid \mapsto \cdot \mapsto q\} \in \txMint : q < 0$.
\end{menumerate}

\noindent The state-machine validator $\nuHead$ is spent with
$\redeemerHead = (\mathsf{abort}, m)$, where $m$ is the number of outputs to
reimburse, and checks:
\begin{menumerate}
  \item State is advanced from $\datumHead \sim \stInitial$ to terminal state
  $\stFinal$:
  \[
	(\stInitial,\cid,\seed,\hydraKeys,\Tcontest) \xrightarrow[m]{\stAbort} \stFinal.
  \]
  \item All UTxOs committed into the head are reimbursed exactly as they were
  committed. This is done by comparing hashes of serialised representations of
  the $m$ reimbursing outputs $o_{1} \dots o_{m}$\footnote{Only the first $m$
	outputs are used for reimbursing, while more outputs may be present in the
	transaction, e.g for returning change.} with the canonically combined
  committed UTxOs in $C_{i}$:
  \[
	\hash(\bigoplus_{j=1}^{m} \bytes(o_{j})) = \combine([C_{i} ~ | ~ \forall [1 \dots |\hydraKeys|], C_{i} \neq \bot])
  \]

  \item Transaction is signed by a participant $\exists \{\cid \mapsto \keyHash_{i} \mapsto -1\} \in \txMint \Rightarrow \keyHash_{i} \in \txKeys$.
  \item All tokens are burnt
  $|\{\cid \mapsto \cdot \mapsto -1\} \in \txMint| = |\hydraKeys| + 1$.
\end{menumerate}

\begin{figure}
  \centering
  \includesvg[width=0.8\textwidth]{Hydra/Protocol/Figures/abortTx.svg}
  \caption{\mtxAbort{} transaction spending the $\stInitial$ state head
	output and collecting all initial and commit outputs, which get reimbursed
	by outputs $o_{1} \dots o_{m}$. Note that each $\pt$ may be in either, an
	initial or commit output.}\label{fig:abortTx}
\end{figure}

\subsection{CollectCom Transaction}\label{sec:collect-tx}

\noindent The \mtxCCom{} transaction (Figure~\ref{fig:collectComTx}) collects
all the committed UTxOs to the same head. It has
\begin{itemize}
  \item one input spending from $\nuHead$ holding the $\st$ with $\datumHead$,
  \item $\forall i \in \{1 \dots |\hydraKeys|\}$ inputs spending commit outputs
  $(\valCommit{i}, \nuCommit, \datumCommit{i})$ with $\pt_{i} \in \valCommit{i}$
  and $\datumCommit{i} = (\cid, C_{i})$, and
  \item one output paying to $\nuHead$ with value $\valHead'$ and
  datum $\datumHead'$.
\end{itemize}

\noindent The state-machine validator $\nuHead$ is spent with
$\redeemerHead = \mathsf{collect}$ and checks:
\begin{menumerate}
  \item State is advanced from $\datumHead \sim \stInitial$ to
  $\datumHead' \sim \stOpen$, parameters $\cid,\hydraKeys,\Tcontest$ stay
  unchanged and the new state is governed again by $\nuHead$
  \[
	(\stInitial,\cid,\seed,\hydraKeys,\Tcontest) \xrightarrow{\stCollect} (\stOpen,\cid,\hydraKeys,\Tcontest,v,\eta)
  \]
  where snapshot version is initialized as $v = 0$.
  \item Commits are collected in $\eta$
  \[
	\eta = U^{\#} = \combine([C_{1}, \dots, C_{n}])
  \]
  where $n = |\hydraKeys|$ and
  \[
	\combine(\underline{C}) = \hash(\mathsf{concat}({\sortOn(1, \mathsf{concat}(\underline{C}))}^{\downarrow2}))
  \]
  That is, given a list of committed UTxO $\underline{C}$, where each element is
  a list of output references and the serialised representation of what was
  committed, $\combine$ first concatenates all commits together, sorts this list
  by the output references, concatenates all bytes and hashes the
  result\footnote{Sorting is required to ensure a canonical representation which
	can also be reproduced from the UTxO set later in the fanout.}.

  \item All committed value captured and no value is extracted
  \[
	\valHead' = \valHead \cup (\bigcup_{i=1}^{n} \valCommit{i})
  \]
  \item Every participant had the chance to commit, by checking all tokens are
  present in output\footnote{This is sufficient as a Head participant would
	check off-chain whether a Head is initialized correctly with the right
	number of tokens.}
  % NOTE: Implemented slightly different as we would only count PTs and = n
  \[
	|\{\cid \rightarrow . \rightarrow 1\} \in \valHead'| = |\hydraKeys| + 1
  \]
  \item Transaction is signed by a participant
  \[
	\exists \{\cid \mapsto \keyHash_{i} \mapsto 1\} \in \valCommit{i} \Rightarrow \keyHash_{i} \in \txKeys
  \]
  \item No minting or burning
  \[
	\txMint = \varnothing
  \]
\end{menumerate}

\noindent Each spent $\nuCommit$ validator with $\datumCommit{i} = (\cid,\cdot)$ and $\redeemerCommit{i} = \mathsf{collect}$ ensures that:
\begin{menumerate}
  \item The state token of currency $\cid$ is present in the output value
  \[
	\st \in \valHead'
  \]
\end{menumerate}

\begin{figure}
  \centering
  \includesvg[width=0.8\textwidth]{Hydra/Protocol/Figures/collectComTx.svg}
  \caption{\mtxCCom{} transaction spending the head output in $\stInitial$
	state and collecting from multiple commit outputs into a single
	$\stOpen$ head output.}\label{fig:collectComTx}
\end{figure}


\subsection{Deposit Transaction}\label{sec:deposit-tx}

\noindent The \mtxDeposit{} transaction initiates a (incremental) commit by
locking funds in $\nuDeposit$ for later collection by the head protocol. Any
transaction paying to $\nuDeposit$ is a \mtxDeposit{} transaction as there is no
on-chain verification in \mtxDeposit{} transactions. Consequently, protocol
actors \textbf{must ensure off-chain} that a valid datum is used when paying to
the $\nuDeposit$ validator.\todo{explain why this is enough?} \\

\noindent A valid deposit output is governed by $\nuDeposit$ with value $\valDeposit$ and datum
\[
  \datumDeposit = (\cid, t_{\mathsf{recover}}, C)
\]
where
\begin{mitemize}
  \item $\cid$ is the currency id of the target head protocol instance (see~\ref{sec:init-tx}),
  \item $t_{\mathsf{recover}}$ is a deadline after which the deposit can be recovered, and
  \item $C \in {(\txInputs \times \tyBytes)}^{*}$ is a list of transaction output
  references with along with serialized outputs that should become available in
  the head (similar to commits in~\ref{sec:commit-tx}).
\end{mitemize}

\noindent Head protocol participants determine \textbf{off-chain} whether a
deposit output $o_{\mathsf{desposit}}$ is eligible for their head by checking
\begin{enumerate}
  \item $\cid$ matches their head identifier,
  \item $t_{\mathsf{recover}}$ is reasonably far in the future, and\todo{explain; relate to contestation period?}
  \item $\valDeposit$ contains the value of all decoded outputs of $C$ from $\datumDeposit$.
\end{enumerate}

\noindent An example transaction which records all its spent inputs in a deposit output is
shown in Figure~\ref{fig:depositTx}. The $j \in \{1 \dots m\}$ inputs of this example with reference $\txOutRef_{\mathsf{deposited}_{j}}$ each spend output $o_{\mathsf{deposited}_{j}}$ with $\val_{\mathsf{deposited}_{j}}$ would be recorded in the output datum as
\[
  C = \forall j \in \{1 \dots m\} : [(\txOutRef_{\mathsf{deposited}_{j}},\bytes(o_{\mathsf{deposited}_{j}}))]
\]
\noindent and the value check would need to satisfy
\[
  \valDeposit \supseteq \bigcup_{j=1}^{m} \val_{\mathsf{deposited}_{j}}
\]
\begin{figure}
  \centering
  \includesvg[width=0.8\textwidth]{Hydra/Protocol/Figures/depositTx.svg}
  \caption{\mtxDeposit{} transaction spending multiple UTxO into a deposit
	output.}\label{fig:depositTx}
\end{figure}

\subsection{Recover Transaction}\label{sec:recover-tx}

\noindent If a \mtxDeposit{} transaction output (see~\ref{sec:deposit-tx}) was
not collected into a head by an \mtxIncrement{}
transaction~\ref{sec:increment-tx}, the \mtxRecover{} transaction
(Figure~\ref{fig:recoverTx}) allows for restoring the UTxO as recorded in the
deposit after the deadline has passed. It consists of
\begin{itemize}
  \item one input spending from $\nuDeposit$ with datum $\datumDeposit = (\cid, t_{\mathsf{recover}}, C)$, and
  \item outputs $o_{1} \dots o_{m}$ to recover UTxOs.
\end{itemize}

\noindent The script validator $\nuDeposit$ is spent with redeemer
$\redeemerDeposit = (\mathsf{Recover}, m)$, where $m$ is the number of outputs
to recover, and checks:
\begin{menumerate}
  \item All UTxOs are recovered exactly as they were deposited. This is done by
  comparing hashes of serialised representations of the $m$ recovering outputs
  $o_{1} \dots o_{m}$ with the canonically combined committed UTxOs in $C$:
  \[
    \hash(\bigoplus_{j=1}^{m} \bytes(o_{j})) = \hash(\mathsf{concat}({\sortOn(1, C)}^{\downarrow2}))
  \]
  \item Transaction is posted after the deadline
  \[
    \txValidityMin > t_{\mathsf{recover}}
  \]
\end{menumerate}

\begin{figure}
  \centering
  \includesvg[width=0.8\textwidth]{Hydra/Protocol/Figures/recoverTx.svg}
  \caption{\mtxRecover{} transaction restoring UTxO of a deposit
	output.}\label{fig:recoverTx}
\end{figure}

\subsection{Increment Transaction}\label{sec:increment-tx}

\noindent The \mtxIncrement{} transaction (Figure~\ref{fig:incrementTx}) allows
a participant to add a \mtxDeposit{} output~\ref{sec:deposit-tx} to an already
open head using a snapshot that approves inclusion. Consequently this
transaction consists of:

\begin{itemize}
  \item one input spending from $\nuHead$ with value $\valHead$ holding the
        $\st$ and datum $\datumHead$,
  \item one input $\txOutRef_{\mathsf{deposit}}$ spending from $\nuDeposit$ with value $\valDeposit$ and datum
        $\datumDeposit = (\cid_{\mathsf{deposit}}, t_{\mathsf{recover}}, C)$,
  \item one output paying to $\nuHead$ with value $\valHead'$ and datum
        $\datumHead'$.
\end{itemize}

\noindent The deposit validator $\nuDeposit$ is spent with
$\redeemerDeposit = (\mathsf{claim}, \mathsf{\cid})$ and ensures:
\begin{menumerate}
  \item Claiming head id matches the deposit datum
  \[
    \cid = \cid_{\mathsf{deposit}}
  \]
  \item Transaction is posted before the deadline
  \[
    \txValidityMax <= t_{\mathsf{recover}}
  \]
\end{menumerate}

\noindent The state-machine validator $\nuHead$ is spent with
$\redeemerHead = (\mathsf{increment}, \xi, s, \txOutRef_{\mathsf{increment}})$,
where $\xi$ is a multi-signature of the increment snapshot which authorizes
addition of deposited UTxO, $s$ is the snapshot number and
$\txOutRef_{\mathsf{deposit}}$ points to the claimed deposit. The validator
checks:
\begin{menumerate}
  \item State is advanced from $\datumHead \sim \stOpen$ to
  $\datumHead' \sim \stOpen$, parameters $\cid,\hydraKeysAgg,\nop,\Tcontest$
  stay unchanged and the new state is governed again by $\nuHead$:
  \[
	(\stOpen,\cid,\hydraKeysAgg,\nop,\Tcontest,v,\eta) \xrightarrow[\xi, s, \txOutRef_{\mathsf{increment}}]{\mathsf{increment}} (\stOpen,\cid,\hydraKeysAgg,\nop,\Tcontest,v',\eta')
  \]
  \item New version $v'$ is incremented correctly
  \[
	v' = v + 1
  \]
  \item Claimed deposit is spent
  \[
    \txOutRef_{\mathsf{increment}} = \txOutRef_{\mathsf{deposit}}
  \]
  \item $\xi$ is a valid multi-signature of the new head state $\eta'$
  \[
	\msVfy(\hydraKeys,(\cid || v || s || \eta' || \eta_\alpha || \bot),\xi) = \true
  \]
  where $\eta_\alpha$ is the digest of all deposited UTxO in $C$ sorted by their output
  references
  \[
	\eta_\alpha = \hash(\mathsf{concat}({\sortOn(1, C)}^{\downarrow2}))
  \]
  \item The value in the head output is increased accordingly\todo{Only check $\valHead' > \valHead$?}
  \[
	\valHead \cup \valDeposit = \valHead'
  \]
  \item Transaction is signed by a participant\todo{Redundant to snapshot sig?}
  \[
	\exists \{\cid \mapsto \keyHash_{i} \mapsto 1\} \in \valHead' \Rightarrow \keyHash_{i} \in \txKeys
  \]
\end{menumerate}


\begin{figure}
  \centering
  \includesvg[width=0.8\textwidth]{Hydra/Protocol/Figures/incrementTx.svg}
  \caption{\mtxIncrement{} transaction spending an open head output,
	producing a new head output which includes the new UTxO.}\label{fig:incrementTx}
\end{figure}

\subsection{Decrement Transaction}\label{sec:decrement-tx}

\noindent The \mtxDecrement{} transaction (Figure~\ref{fig:decrementTx}) allows
a party to remove a UTxO from an already open head and consists of:

\begin{itemize}
  \item one input spending from $\nuHead$ holding the $\st$ with $\datumHead$,
  \item one output paying to $\nuHead$ with value $\valHead'$ and
  datum $\datumHead'$,
  \item one or more decommit outputs $o_{2} \dots o_{m+1}$ with values $\val_{2} \dots \val_{m+1} $.
\end{itemize}

\noindent The state-machine validator $\nuHead$ is spent with
$\redeemerHead = (\mathsf{decrement}, \xi, s, m)$, where $\xi$ is a multi-signature of
the decrement snapshot which authorizes removal of some UTxO, $s$ is the
used snapshot number and $m$ is the number of outputs to distribute. The
validator checks:
\begin{menumerate}
  \item State is advanced from $\datumHead \sim \stOpen$ to
  $\datumHead' \sim \stOpen$, parameters $\cid,\hydraKeys,\Tcontest$ stay
  unchanged and the new state is governed again by $\nuHead$
  \[
	(\stOpen,\cid,\hydraKeys,\Tcontest,v,\eta) \xrightarrow[\xi, s, m]{\mathsf{decrement}} (\stOpen,\cid,\hydraKeys,\Tcontest,v',\eta')
  \]
  \item New version $v'$ is incremented correctly
  \[
	v' = v + 1
  \]
  \item $\xi$ is a valid multi-signature of the new snapshot state $\eta'$
  \[
	\msVfy(\hydraKeys,(\cid || v || s || \eta' || \eta\alpha || \eta_\omega),\xi) = \true
  \]
  where $\eta_\omega$ is the digest of all removed UTxO
  \[
	\eta_\omega = \hash(\bigoplus_{j=2}^{m+1} \bytes(o_{j}))
  \]
  \item The value in the head output is decreased accordingly\todo{Only check $\valHead' < \valHead$?}
  \[
	\valHead' \cup (\bigcup_{j=2}^{m+1} \val_{j}) = \valHead
  \]
  \item Transaction is signed by a participant\todo{Redundant to snapshot sig?}
  \[
	\exists \{\cid \mapsto \keyHash_{i} \mapsto 1\} \in \valHead' \Rightarrow \keyHash_{i} \in \txKeys
  \]
\end{menumerate}

\begin{figure}
  \centering
  \includesvg[width=0.8\textwidth]{Hydra/Protocol/Figures/decrementTx.svg}
  \caption{\mtxDecrement{} transaction spending an open head output,
	producing a new head output and multiple decommitted outputs.}\label{fig:decrementTx}
\end{figure}

\subsection{Close Transaction}\label{sec:close-tx}

In order to close a head, a head member may post the \mtxClose{} transaction
(see Figure~\ref{fig:closeTx}). This transaction has
\begin{itemize}
  \item one input spending from $\nuHead$ holding the $\st$ with $\datumHead$,
  \item one output paying to $\nuHead$ with value $\valHead'$ and
  datum $\datumHead'$.
\end{itemize}

\begin{figure}
  \centering
  \includesvg[width=0.8\textwidth]{Hydra/Protocol/Figures/closeTx.svg}
  \caption{\mtxClose{} transaction spending the $\stOpen$ head output and producing a $\stClosed$ head output.}\label{fig:closeTx}
\end{figure}

\noindent The state-machine validator $\nuHead$ is spent with
$\redeemerHead = (\mathsf{close}, \mathsf{CloseType})$, where
$\mathsf{CloseType}$ is a hint against which open state to close and checks:
\begin{menumerate}
  \item State is advanced from $\datumHead \sim \stOpen$ to
  $\datumHead' \sim \stClosed$, parameters $\cid,\hydraKeys,\Tcontest$
  stay unchanged and the new state is governed again by $\nuHead$
  \[
	(\stOpen,\cid,\hydraKeys,\Tcontest,v,\eta) \xrightarrow[\mathsf{CloseType}]{\stClose} (\stClosed,\cid,\hydraKeys,\Tcontest,v',s',\eta', \eta_\alpha\Delta', \eta_\omega\Delta', \contesters,\tfinal)
  \]
  \item Last known open state version is recorded in closed state
  \[
	v' = v
  \]

  \item Based on the redeemer $\mathsf{CloseType} = \mathsf{Initial} \cup (\mathsf{Any}, \xi) \cup (\mathsf{UnusedInc}, \xi, \eta_\alpha) \cup (\mathsf{UnusedDec}, \xi) \cup (\mathsf{UsedInc}, \xi)  \cup (\mathsf{UsedDec}, \xi, \eta_\omega) $, where $\xi$ is a multi-signature of the closing snapshot and $\eta_\alpha$ and $\eta_\omega$ are the digests of the UTxO to increment or decrement respectively, six cases are distinguished:
  \begin{menumerate}
	\item $\mathsf{Initial}$: The initial snapshot is used to close the head and open state was not updated. No signatures are available and it suffices to check
	\[
	  v = 0
	\]
	\[
	  s' = 0
	\]
	\[
	  \eta' = \eta
	\]
    \item $\mathsf{Any}$: Closing snapshot refers to current state version $v$ and both UTxO to increment and decrement must be empty in the closed state.
	  \[
		\eta_\alpha\Delta' = \eta_\alpha = \bot
	  \]
	  \[
		\eta_\omega\Delta' = \eta_\omega = \bot
	  \]
	  \[
		\msVfy(\hydraKeys,(\cid || v || s' || \eta' || \eta_\alpha || \eta_\omega),\xi) = \true
	  \]
	  \item $\mathsf{UnusedInc}$: Closing snapshot refers to current state version $v$ and any UTxO to increment must not be recorded in the closed state.
	  \[
	    \eta_\alpha  \neq \bot
	  \]
	  \[
	    \eta_\alpha\Delta' = \bot
	  \]
	  \[
		\eta_\omega\Delta' = \eta\omega = \bot
	  \]
	  \[
		\msVfy(\hydraKeys,(\cid || v || s' || \eta' || \eta_\alpha || \eta_\omega),\xi) = \true
	  \]
	  where $\eta_\alpha$ is provided by the redeemer \footnote{$\eta_\alpha$ needs to be provided to verify the signature, but is otherwise not relevant for the closed state}.
	\item $\mathsf{UnusedDec}$: Closing snapshot refers to current state version $v$ and any UTxO to decrement need to be present in the closed state too.
	\[
	  \eta_\alpha\Delta' = \eta_\alpha = \bot
	\]
	\[
	  \eta_\omega\Delta' = \eta_\omega \neq \bot
	\]
	\[
	  \msVfy(\hydraKeys,(\cid || v || s' || \eta' || \eta_\alpha || \eta_\omega),\xi) = \true
	\]
	\todo{this is hard to understand}
	\item $\mathsf{UsedInc}$: Closing snapshot refers the previous state $v - 1$ and any UTxO to increment must be recorded in the closed state.
	  \[
		\eta_\alpha\Delta' = \eta_\alpha \neq \bot
	  \]
	  \[
	    \eta_\omega\Delta' = \eta_\omega = \bot
	  \]
	  \[
		\msVfy(\hydraKeys,(\cid || v - 1 || s' || \eta' || \eta_\alpha || \eta_\omega ),\xi) = \true
	\]
	\item $\mathsf{UsedDec}$: Closing snapshot refers the previous state $v - 1$ and any UTxO to decrement must not be recorded in the closed state.
	  \[
		\eta_\alpha\Delta' = \eta_\alpha = \bot
	  \]
	  \[ 
	    \eta_\omega \neq  \bot
	  \]
	  \[ 
	    \eta_\omega\Delta' = \bot
	  \]
	\[
	  \msVfy(\hydraKeys,(\cid || v - 1 || s' || \eta' || \eta_\alpha || \eta_\omega),\xi) = \true
	\]
	where $\eta_\omega$ is provided by the redeemer\footnote{$\eta_\omega$ needs to be provided to verify the signature, but is otherwise not relevant for the closed state}.
  \end{menumerate}
  % TODO: detailed CDDL definition of msg

  \item Initializes the set of contesters
  \[
	\contesters = \emptyset
  \]
  This allows the closing party to also contest and is required for use
  cases where pre-signed, valid in the future, close transactions are
  used to delegate head closing.

  \item Correct contestation deadline is set
  \[
	\tfinal = \txValidityMax + T
  \]
  \item Transaction validity range is bounded by
  \[
	\txValidityMax - \txValidityMin \leq T
  \]
  to ensure the contestation deadline $\tfinal$ is at most $2*T$ in the future.
  \item Value in the head is preserved
  \[
	\valHead' = \valHead
  \]
  \item Transaction is signed by a participant
  \[
	\exists \{\cid \mapsto \keyHash_{i} \mapsto 1\} \in \valCommit{i} \Rightarrow \keyHash_{i} \in \txKeys
  \]
  \item No minting or burning
  \[
	\txMint = \varnothing
  \]
\end{menumerate}

\subsection{Contest Transaction}\label{sec:contest-tx}

The \mtxContest{} transaction (see Figure~\ref{fig:contestTx}) is posted by a
party to prove the currently $\stClosed$ state is not the latest one. This
transaction has
\begin{itemize}
  \item one input spending from $\nuHead$ holding the $\st$ with $\datumHead$,
  \item one output paying to $\nuHead$ with value $\valHead'$ and
  datum $\datumHead'$.
\end{itemize}

\begin{figure}
  \includesvg[width=0.8\textwidth]{Hydra/Protocol/Figures/contestTx.svg}
  \caption{\mtxContest{} transaction spending the $\stClosed$ head output and
	producing a different $\stClosed$ head output.}\label{fig:contestTx}
\end{figure}

\noindent The state-machine validator $\nuHead$ is spent with
$\redeemerHead = (\mathsf{contest}, \mathsf{ContestType})$, where
$\mathsf{ContestType}$ is a hint how to verify the snapshot and checks:
\begin{menumerate}
  \item State is advanced from $\datumHead \sim \stOpen$ to
  $\datumHead' \sim \stClosed$, parameters $\cid,\hydraKeys,\Tcontest$
  stay unchanged and the new state is governed again by $\nuHead$
  \[
	(\stClosed,\cid,\hydraKeys,\Tcontest,v,s,\eta,\eta_\alpha\Delta,\eta_\omega\Delta,\contesters,\tfinal) \xrightarrow[\mathsf{ContestType}]{\stContest} (\stClosed,\cid,\hydraKeys,\Tcontest,v',s',\eta',\eta_\alpha\Delta',\eta\omega\Delta',\contesters',\tfinal')
  \]

  \item Last known open state version stays recorded in closed state
  \[
	v' = v
  \]

  \item Contested snapshot number $s'$ is higher than the currently stored snapshot number $s$
  \[
	s' > s
  \]
  \item Based on the redeemer $\mathsf{ContestType} = (\mathsf{Current}, \xi) (\mathsf{UnusedInc}, \xi, \eta_\alpha) \cup (\mathsf{UnusedDec}, \xi) \cup (\mathsf{UsedInc}, \xi)  \cup (\mathsf{UsedDec}, \xi, \eta_\omega)$, where $\xi$ is a multi-signature of the contesting snapshot and $\eta_\alpha$ and $\eta_\omega$ are the digests of the UTxO to increment or decrement respectively, five cases are distinguished:

  \begin{menumerate}
	\item $\mathsf{Current}$: Contesting snapshot refers to current state version $v$ and any UTxO to increment or decrement must be $\bot$ in the closed state.
	  \[
		\eta_\alpha\Delta' = \eta_\alpha =  \bot
	  \]
	  \[
		\eta_\omega\Delta' = \eta_\omega =  \bot
	  \]
	  \[
		\msVfy(\hydraKeys,(\cid || v || s' || \eta' || \eta_\alpha || \eta_\omega),\xi) = \true
	  \]
	%% FIXME: \red{\item ...} does not work
	\item $\mathsf{UnusedInc}$: Contesting snapshot refers to current state version $v$ and any UTxO to increment must be $\bot$ in the closed state.
	  \[
	    \eta_\alpha  \neq \bot
	  \]
	  \[
	    \eta_\alpha\Delta' = \bot
	  \]
	  \[
		\eta_\omega\Delta' = \eta\omega = \bot
	  \]
	  \[
		\msVfy(\hydraKeys,(\cid || v || s' || \eta' || \eta_\alpha || \eta_\omega),\xi) = \true
	  \]
	  where $\eta_\alpha$ is provided by the redeemer \footnote{$\eta_\alpha$ needs to be provided to verify the signature, but is otherwise not relevant for the closed state}.

	\item $\mathsf{UnusedDec}$: Contesting snapshot refers to current state version $v$ and any UTxO to decrement need to be present in the closed state too.
	\[
	  \eta_\alpha\Delta' = \eta_\alpha = \bot
	\]
	\[
	  \eta_\omega\Delta' = \eta_\omega \neq \bot
	\]
	\[
	  \msVfy(\hydraKeys,(\cid || v || s' || \eta' || \eta_\alpha|| \eta_\omega),\xi) = \true
	\]
	  \item $\mathsf{UsedInc}$: Contesting snapshot refers the previous state $v - 1$ and any UTxO to increment must be recorded in the closed state.
	  \[
		\eta_\alpha\Delta' = \eta_\alpha \neq \bot
	  \]
	  \[
	    \eta_\omega\Delta' = \eta_\omega = \bot
	  \]
	  \[
		\msVfy(\hydraKeys,(\cid || v - 1 || s' || \eta' || \eta_\alpha || \eta_\omega ),\xi) = \true
	  \]

	\item $\mathsf{UsedDec}$: Contesting snapshot refers the previous state $v - 1$ and any UTxO to decrement must not be recorded in the closed state.
	  \[
		\eta_\alpha\Delta' = \eta_\alpha = \bot
	  \]
	  \[ 
	    \eta_\omega \neq  \bot
	  \]
	  \[ 
	    \eta_\omega\Delta' = \bot
	  \]
	\[
	  \msVfy(\hydraKeys,(\cid || v - 1 || s' || \eta' || \eta_\alpha|| \eta_\omega),\xi) = \true
	\]
	where $\eta_\omega$ is provided by the redeemer\footnote{$\eta_\omega$ needs to be provided to verify the signature, but is otherwise not relevant for the closed state}.
  \end{menumerate}
  % TODO: detailed CDDL definition of msg

  \item The single signer $\{\keyHash\} = \txKeys$ has not already contested and is added to the set of contesters
  \[
	\keyHash \notin \contesters
  \]
  \[
	\contesters' = \contesters \cup \keyHash
  \]
  \item Transaction is posted before deadline
  \[
	\txValidityMax \leq \tfinal
  \]
  \item Contestation deadline is updated correctly to
  \[
	\tfinal' = \left\{\begin{array}{ll}
	  \tfinal     & \mathrm{if} ~ |\contesters'| = n, \\
	  \tfinal + T & \mathrm{otherwise.}
	\end{array}\right.
\]
\item Value in the head is preserved
\[
  \valHead' = \valHead
\]
\item Transaction is signed by a participant
\[
  \exists \{\cid \mapsto \keyHash_{i} \mapsto 1\} \in \valCommit{i} \Rightarrow \keyHash_{i} \in \txKeys
\]
\item No minting or burning
\[
  \txMint = \varnothing
\]
\end{menumerate}

\subsection{Fan-Out Transaction}\label{sec:fanout-tx}

Once the contestation phase is over, a head may be finalized by posting a
\mtxFanout{} transaction (see Figure~\ref{fig:fanoutTx}), which
distributes UTxOs from the head according to the latest state. It consists of
\begin{itemize}
  \item one input spending from $\nuHead$ holding the $\st$, and
  \item outputs $o_{1} \dots o_{m+n+n'}$ to distribute UTxOs.
\end{itemize}

Note that \mtxFanout{} represents a final transition of the state machine and
hence there is no state machine output.

\begin{figure}
  \centering
  \includesvg[width=0.8\textwidth]{Hydra/Protocol/Figures/fanoutTx.svg}
  \caption{\mtxFanout{} transaction spending the $\stClosed$ head output and
	distributing funds with outputs $o_{1} \dots o_{m+n+n'}$.}\label{fig:fanoutTx}
\end{figure}

\noindent The state-machine validator $\nuHead$ is spent with
$\redeemerHead = (\mathsf{fanout}, m, n, n')$, where $m$, $n$ and $n'$ are
outputs to distribute from the $\stClosed$ state, and checks:
\begin{menumerate}
  \item State is advanced from $\datumHead \sim \stClosed$ to terminal state
  $\stFinal$: % XXX: What does this actually mean?
  \[
	(\stClosed,\cid,\hydraKeys,\Tcontest,v, s,\eta,\eta_\alpha\Delta,\eta_\omega\Delta,\contesters,\tfinal) \xrightarrow[m,n,n']{\stFanout} \stFinal
  \]
  \item The first $m$ outputs are distributing funds according to $\eta$. That is,
  the outputs exactly correspond to the UTxO canonically combined $U^{\#}$ (see
  Section~\ref{sec:collect-tx}):
  \[
	\eta = U^{\#} = \hash(\bigoplus_{j=1}^{m} \bytes(o_{j}))
  \]
  \item The following $n$ outputs are distributing funds according to
  $\eta_\alpha\Delta$. That is, the outputs exactly correspond to the UTxO canonically
  combined $U^{\#}_{\alpha\Delta}$ (see Section~\ref{sec:collect-tx}):
  \[
	\eta_{\alpha\Delta} = U^{\#}_{\alpha\Delta} = \hash(\bigoplus_{j=m}^{m+n} \bytes(o_{j}))
  \]
  \item The next $n'$ outputs are distributing funds according to
  $\eta_\omega\Delta$. That is, the outputs exactly correspond to the UTxO canonically
  combined $U^{\#}_{\omega\Delta}$ (see Section~\ref{sec:collect-tx}):
  \[
	\eta_{\omega\Delta} = U^{\#}_{\omega\Delta} = \hash(\bigoplus_{j=m'}^{m+n'} \bytes(o_{j}))
  \]
  \item Transaction is posted after contestation deadline $\txValidityMin > \tfinal$.
  \item All tokens are burnt
  $|\{\cid \mapsto \cdot \mapsto -1\} \in \txMint| = n + 1$.
\end{menumerate}

\noindent The $\muHead(\seed)$ minting policy governs the burning of tokens via
redeemer $\mathsf{burn}$ that:
\begin{menumerate}
  \item All tokens in $\txMint$ need to be of negative quantity
  $\forall \{\cid \mapsto \cdot \mapsto q\} \in \txMint : q < 0$.
\end{menumerate}

\FloatBarrier{}

%%% Local Variables:
%%% mode: latex
%%% TeX-master: "main"
%%% End:
