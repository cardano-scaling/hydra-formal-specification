\section{Off-Chain Protocol}\label{sec:offchain}

This section describes the actual Coordinated Hydra Head protocol, an even more
simplified version of the original publication~\cite{hydrahead20}. See the
protocol overview in Section~\ref{sec:overview} for an introduction and notable
changes to the original protocol. While the on-chain part already describes the
full life-cycle of a Hydra head on-chain, this section completes the picture by
defining how the protocol behaves off-chain and notably the relationship between
on- and off-chain semantics. Participants of the protocol are also called Hydra
head members, parties or simply protocol actors. The protocol is specified as a
reactive system that processes three kinds of inputs:
\begin{enumerate}
  \item On-chain protocol transactions as introduced in
        Section~\ref{sec:on-chain}, which are posted to the mainchain and can be
        observed by all actors:
        \begin{itemize}
          \item $\mathtt{initialTx}$: initiates a head
          \item $\mathtt{commitTx}$: commits UTxO to an initializing head
          \item $\mathtt{collectComTx}$: opens a head
          \item $\mathtt{depositTx}$: some UTxO was deposited to be incremented
          \item $\mathtt{recoverTx}$: deposited UTxO was recovered
          \item $\mathtt{incrementTx}$: adds UTxO to an open head
          \item $\mathtt{decrementTx}$: removes UTxO from an open head
          \item $\mathtt{closeTx}$: closes a head
          \item $\mathtt{contestTx}$: contests a closed head
          % NOTE: fanout not mentioned because not needed in off-chain protocol
          % description
        \end{itemize}
        Also, a special input when time advanced on chain may be used:
        \begin{itemize}
          \item $\mathtt{tick}$: time advanced on chain
        \end{itemize}
  
  \item Off-chain network messages sent between protocol actors (parties):
  \begin{itemize}
    \item $\hpRT$: to request a transaction to be included in the next snapshot
    \item $\hpRD$: to request exclusion of UTxO via a decommit transaction
    \item $\hpRS$: to request a snapshot to be created \& signed by every head member
    \item $\hpAS$: to acknowledge a snapshot by replying with their signatures
  \end{itemize}
  \item Commands issued by the participants themselves or on behalf of end-users and clients
  \begin{itemize}
    \item $\hpInit$: to start initialization of a head
    \item $\hpClose$: to request closure of an open head
  \end{itemize}
\end{enumerate}

% TODO: define states and e.g. that newTx not possible when closed? state diagram?

The behavior is fully specified in Figure~\ref{fig:off-chain-prot}, while the
following paragraphs introduce notation, explain variables and walk-through the
protocol flow.

\subsection{Assumptions}

On top of the statements of the protocol setup in Section~\ref{sec:setup}, the
off-chain protocol logic relies on these assumptions:
\todo{move/merge with protocol setup?}
\begin{itemize}
  \item Every network message received from a specific party is checked for
        authentication. An implementation of the specification needs to find a
        suitable means of authentication, either on the communication channel or
        for individual messages. Unauthenticated messages must be dropped.
  \item The head protocol gets correctly (and with completeness) notified about
        observed transactions on-chain belonging to the respective head
        instance.
  \item All inputs are processed to completion, i.e.\ run-to-completion
        semantics and no preemption.
  \item Inputs are deduplicated. That is, any two identical inputs must not lead
        to multiple invocations of the handling semantics.
  \item Given the specification, inputs may pile up forever and implementations
        need to consider these situations (i.e.\ potential for DoS). A valid
        reaction to this would be to just drop these inputs. Note that, from a
        security standpoint, these situations are identical to a
        non-collaborative peer and closing the head is also a possible reaction.
  \item The lifecycle of a Hydra head on-chain does not cross (hard fork)
        protocol update boundaries. Note that these inputs are announced in
        advance hence it should be possible for implementations to react in such
        a way as to expedite closing of the head before such a protocol update.
        This further assumes that the contestation period parameter is picked
        accordingly.
\end{itemize}

\subsection{Notation}
\todo{missing:, apply tx}
\begin{itemize}
  \item $\KwOn~event$ specifies how the protocol reacts on a given input
        $event$. Further information may be available from the constituents of
        $event$ and origin of the input.
  \item $\Req~p$ means that boolean expression $p \in \tyBool$ must be satisfied
        for the further execution of a routine, while discontinued on $\neg p$. A
        conservative protocol actor could interpret this as a reason to close
        the head.
  \item $\KwWait~p$ is a non-blocking wait for boolean predicate $p \in \tyBool$
        to be satisfied. On $\neg p$, the execution of the routine is stopped,
        queued, and reactivated at latest when $p$ is satisfied.
  \item $\Multi{}~msg$ means that a message $msg$ is (channel-) authenticated
        and sent to all participants of this head, including the sender.
  \item $\PostTx{}~tx$ has a party create transaction $tx$, potentially from
        some data, and submit it on-chain. See Section~\ref{sec:on-chain} for
        individual transaction details.
  \item $\Out{}~event$ signals an observation of $event$, which is used in the
        security definition and proofs of Section~\ref{sec:security}. This
        keyword can be ignored when implementing the protocol.
\end{itemize}

\subsection{Variables}

Besides parameters agreed in the protocol setup (see Section~\ref{sec:setup}), a
party's local state consists of the following variables:

\begin{itemize}
  \item $\hatv$: Last seen open state version.
  \item $\hats$: Sequence number of latest seen snapshot.
  \item $\hatSigma \in {(\tyNatural \times \tyBytes)}^{*}$: Accumulator of
        signatures of the latest seen snapshot, indexed by parties.
  \item $\hatmL$: UTxO set representing the local ledger state resulting from
        applying $\hatmT$ to $\bar{S}.U$ to validate requested transactions.
  \item $\hatmT \in \mT^{*}$: List of transactions applied locally and pending
        inclusion in a snapshot (if this party is the next leader).
  \item $\tx_\alpha \in \mathcal{T}$: Pending deposit transaction\footnote{In
        fact this would only need to be a transaction id to look up the
        corresponding deposit in $\mc D$} to be used for incrementing the head
        state.
  \item $\tx_\omega \in \mathcal{T}$: Pending decrement transaction, whose outputs are to be
        withdrawn from the head.
  \item $\mc D$: Set of deposit objects tracking deposit transactions with their
        status.
  \item $\bar{\mc S}$: Snapshot object of the latest confirmed snapshot which
        contains:
        \begin{center}
          \begin{tabular}{|l|l|}\hline
            $\bar{\mc S}.v$         & snapshot version \\ \hline
            $\bar{\mc S}.s$         & snapshot number \\ \hline
            $\bar{\mc S}.\mT$       & list of transactions relating this snapshot to the previous \\ \hline
            $\bar{\mc S}.U$         & snapshotted UTxO set \\ \hline
            $\bar{\mc S}.U_\alpha$ & pending UTxO to increment \\ \hline
            $\bar{\mc S}.U_\omega$       & pending UTxO to decrement \\ \hline
            $\bar{\mc S}.\sigma$         & multisignature \\ \hline
          \end{tabular}
        \end{center}
\end{itemize}

where constructor $\text{snObj}(v, n, T, U, U_\alpha, U_\omega)$ initializes a
new snapshot object with $\bar{\mathcal{S}}.\sigma = \emptyset$. \\

Additionally, deposit objects are created using
$\text{depositObj}(U, t_{\text{created}}, t_{\text{deadline}}, \text{status})$
where status can be $\text{Inactive}$, $\text{Active}$, or $\text{Expired}$.

\subsection{Protocol flow}

\subsubsection{Initializing the head}

\dparagraph{$\hpInit$.}\quad Before a head can be initialized, all parties need
to exchange and agree on protocol parameters during the protocol setup phase
(see Section~\ref{sec:setup}), so we can assume the public Cardano keys
$\cardanoKeys^{setup}$, Hydra keys $\hydraKeysAgg^{setup}$, as well as the
contestation period $\Tcontest^{setup}$ are available. One of the clients then
can start head initialization using the $\hpInit$ command, which will result in
an $\mtxInit$ transaction being posted. Not strictly a protocol parameter, but
the deposit period $\Tdeposit$ would also be made available from
configuration.\\

\dparagraph{$\mathtt{initialTx}$.}\quad All parties will receive this $\mtxInit$
transaction and validate announced parameters against the pre-agreed $setup$
parameters, as well as the structure of the transaction and the minting policy
used. This is a vital step to ensure the initialized Head is valid, which cannot
be checked completely on-chain (see also Section~\ref{sec:init-tx}). \\

\dparagraph{$\mathtt{commitTx}$.}\quad As each party $p_{j}$ posts a
$\mtxCommit$ transaction, the protocol records observed committed UTxOs of each
party $C_j$. With all committed UTxOs known, the $\eta$-state is created (as
defined in Section~\ref{sec:collect-tx}) and the $\mtxCollect$ transaction is
posted. Note that while each participant may post this transaction, only one of
them will be included in the blockchain as the mainchain ledger prevents double
spending. Should any party want to abort, they would post an $\mtxAbort$
transaction and the protocol would end at this point.\\

\dparagraph{$\mathtt{collectComTx}$.}\quad Upon observing the $\mtxCollect$
transaction, the parties compute $\Uinit \gets \bigcup_{j=1}^{n} C_j$ using
previously observed $C_j$ and initialize $\hatmL = \Uinit$. The seen transaction
set is initialized empty $\hatmT = \emptyset$, seen head state version
$\hatv = 0$, as well as snapshot number $\hats = 0$. No deposit transaction
$\tx_{\alpha} = \bot$ and no decrement transaction $\tx_{\omega} = \bot$ are
pending, and the last confirmed snapshot is initialized accordingly
$\bar{\mc S} \gets \blue{\Sno(0, 0, [], \Uinit, \emptyset, \emptyset)}$.

\subsubsection{Processing transactions off-chain}

Transactions are announced and captured in so-called snapshots. Parties generate
snapshots in a strictly sequential round-robin manner. The party responsible for
issuing the $\ith i$ snapshot is the \emph{leader} of the $\ith i$ snapshot.
Leader selection is round-robin per the $\hydraKeys$ from the protocol setup.
While the frequency of snapshots in the general Head protocol~\cite{hydrahead20}
was configurable, the Coordinated Head protocol does specify a snapshot to be
created after each transaction.\\

\dparagraph{$\hpRT$.}\quad Upon receiving request $(\hpRT,\tx)$, the transaction
is applied to the \emph{local} ledger state $\hatmL \applytx \tx$. If not
applicable yet, the protocol does $\KwWait$ to retry later or eventually marks
this transaction as invalid (see assumption about events piling up). After
applying and if there is no current snapshot in flight ($\hats = \bar{\mc S}.s$)
and the receiving party $\party_{i}$ is the next snapshot leader, a message to
request snapshot signatures $\hpRS$ is sent. \\

\dparagraph{$\hpRD$.}\quad Upon receiving request $(\hpRD,\tx_\omega)$, the
transaction is checked against the \emph{local} ledger state and if it is not
applicable yet or another commit or decommit is pending still, the protocol does
$\KwWait$ to retry later or eventually marks the decommit as invalid. After
applying $\tx$, its outputs are removed from \emph{local} ledger state $\hatmL$
so that they are not available any more and the decommit transaction is kept in
the local state ($\tx_\omega$). If there is no current snapshot in flight
($\hats = \bar{\mc S}.s$) and the receiving party $\party_{i}$ is the next
snapshot leader, a message to request snapshot signatures $\hpRS$ containing the
decrement transaction $\tx_\omega$ is sent. \\

\dparagraph{$\mathtt{depositTx}$.}\quad Upon observing a deposit transaction,
each party records the deposit index by deposit transaction id to their local
deposit registry $\mathcal{D}$ with status $\text{Inactive}$. The deposit
contains deposited UTxO $U$, creation time $t_{\text{created}}$, and deadline
$t_{\text{deadline}}$. \\

\dparagraph{$\mathtt{recoverTx}$.}\quad Upon observing a recover transaction,
each party drops the corresponding entry from its deposit registry
$\mathcal{D}$. \\

\dparagraph{$\mathtt{tick}$.}\quad Whenever time advances (on-chain) to point
$t$, parties update the status of deposits in $\mathcal{D}$ accordingly using
the configured deposit period $\Tdeposit$:
\begin{itemize}
  \item $\mathsf{Expired}$ when deadline passed (or too soon): $t > t_{\text{deadline}} - \Tdeposit$
  \item $\mathsf{Active}$ when deposit settled enougth: $t > t_{\text{created}} + \Tdeposit$
\end{itemize}
When deposits become $\mathsf{Active}$ and no other commit / decommit is
pending, and the party is the next snapshot leader, it may request a new
snapshot including the deposit transaction $\tx_{\alpha}$. \\

\dparagraph{$\hpRS$.}\quad Upon receiving request
$(\hpRS,v,s,\underline{\tx}_{\mathsf{req}}, \tx_\alpha, \tx_\omega)$\footnote{Snapshot
  requests with only transaction identifiers and output references are possible
  if all parties keep an index of previously seen transactions and their
  identifiers.} from party $\party_j$, the receiving $\party_i$ $\Req$s that
only a commit or decommit may be pending, and that $v$ refers to the current
open state version, $s$ is the next snapshot number and that party $\party_j$ is
responsible for leading its creation. Party $\party_i$ may have to wait until
the previous snapshot is confirmed ($\bar{\mathcal{S}}.s = \hat{s}$).
Furthermore, the protocol validates the snapshot request by:
\begin{enumerate}
  \item If a decommit is requested: verify the transaction is applicable to the
        last confirmed UTxO set and update the active utxo set with it
  \item If a deposit/commit is requested: verify the corresponding deposit is
        $\text{Active}$ and update the active utxo set with it
  \item If we are on the same version as the last snapshot, any requested
        decommit or commit must match the last snapshot.
  \item Verify all requested transactions $\underline{\tx}_{\mathsf{req}}$ are
        applicable to the active UTxO set
\end{enumerate}
Only then, $\party_i$ increments their seen-snapshot counter $\hats$, resets the
signature accumulator $\hatSigma$, and computes the UTxO set of the new local
snapshot as
$U \gets U_{\mathsf{active}} \applytx \underline{\tx}_{\mathsf{req}}$. Then,
$\party_i$ creates a signature $\msSig_i$ using their signing key
$\hydraSigningKey$ on a message comprised by the $\cid$, the new snapshot number
$\hats$, the new $\eta$ resulting from canonically combining $U$ (see
Section~\ref{sec:close-tx} for details), and either $\eta_{\alpha}$ or
$\eta_{\omega}$ derived from deposited $U_{\alpha}$ or decommit transaction
$\tx_{\omega}$ respectively. The signature is sent to all head members via
message $(\hpAS,\hats,\msSig_i)$. Finally, the local ledger state $\hatmL$ and
pending transaction set $\hatmT$ get pruned by re-applying all locally pending
transactions $\hatmT$ to the just requested snapshot's UTxO set iteratively and
ultimately yielding a ``pruned'' version of $\hatmT$ and $\hatmL$. \\

\dparagraph{$\hpAS$.}\quad Upon receiving acknowledgment $(\hpAS,s,\msSig_j)$, all
participants $\Req$ that it is from an expected snapshot (either the last seen
$\hats$ or + 1), potentially $\KwWait$ for the corresponding $\hpRS$ such that
$\hats = s$ and $\Req$ that the signature is not yet included in $\hatSigma$.
They store the received signature in the signature accumulator $\hatSigma$, and
if the signature from each party has been collected, $\party_i$ aggregates the
multisignature $\msCSig$ and $\Req$ it to be valid (constructing the signed
message as in $\hpRS$). If everything is fine, the snapshot can be considered
confirmed by creating the snapshot object
$\bar{\mc S} \gets \Sno(\hatv, \hats, \hatmT, \hatmU, U_{\alpha}, \mathsf{outputs}(\tx_{\omega}))$
and storing the multi-signature $\msCSig$ in it for later reference. In case
there is a pending decommit, any participant can now submit a \mtxDecrement{}
transaction by providing the just confirmed snapshot with its digests of the
active UTxO set $\eta$ and the to be removed UTxO set $\eta_{\omega}$. If, however, there
was a pending commit, any participant can now submit an $\mathtt{incrementTx}$
by providing the confirmed snapshot with its digests of the active UTxO set $\eta$
and the UTxO set to be added $\eta_\alpha$. Lastly, if $\party_i$ is the next snapshot
leader and there are already transactions to snapshot in $\hatmT$, a
corresponding $\hpRS$ is distributed. \\

\dparagraph{$\mathtt{decrementTx}$.}\quad Upon observing the \mtxDecrement{}
transaction, which removed outputs $U$ from the head, the corresponding pending
decrement transaction is cleared and the observed version $v$ is used for future
snapshots by setting $\hatv \gets v$. Note that the version of the open head state
is incremented on each \mtxDecrement{} transaction as described in
Section~\ref{sec:decrement-tx}. \\

\dparagraph{$\mathtt{incrementTx}$.}\quad Upon observing the \mtxIncrement{}
  transaction, which added outputs $U$ to the head, the local ledger state
  $\hatmL$ is extended with the newly addded UTxO while the pending increment
  state $U_{\alpha}$ is cleared. Also the observed version $v$ is used for future
  snapshots by setting $\hatv = v$. Note that the version of the open head state
  is incremented on each \mtxIncrement{} transaction as described in
  Section~\ref{sec:increment-tx}

\subsubsection{Closing the head}

\dparagraph{$\hpClose$.}\quad In order to close a head, a client issues the
$\hpClose$ input which uses the latest confirmed snapshot $\bar{\mc S}$ to
create the new $\eta$-state from the last confirmed UTxO set, the digest of
either increment or decrement UTxO set ($\eta_\alpha$ or $\eta_\omega$), and the certifiate
$\xi$ using the corresponding multi-signature. With these, the $\mtxClose$ transaction
can be constructed and posted. See Section~\ref{sec:close-tx} for details about this
transaction. \\

\dparagraph{$\mathtt{closeTx}/\mathtt{contestTx}$.}\quad When a party observes
the head getting closed or contested, the $\eta$-state extracted from the
\mtxClose{} or \mtxContest{} transaction represents the latest head status that
has been aggregated on-chain so far (by a sequence of \mtxClose{} and
\mtxContest{} transactions). If the last confirmed (off-chain) snapshot is newer
than the observed (on-chain) snapshot number $s_{c}$, an updated $\eta$-state,
along with the digest of either increment or decrement UTxO set ($\eta_\alpha$ or $\eta_\omega$),
and certificate $\xi$ is constructed and posted in a \mtxContest{} transaction (see
Section~\ref{sec:contest-tx}).

\subsection{Rollbacks and protocol changes}\label{sec:rollbacks}
\todo{Explain why rollbacks are no problem to increment/decrement}
\todo{Write about contestation deadline vs. rollbacks}

The overall life-cycle of the Head protocol is driven by on-chain inputs (see
introduction of Section~\ref{sec:offchain}) which stem from observing
transactions on the mainchain. Most blockchains, however, do only provide
\emph{eventual} consistency. The consensus algorithm ensures a consistent view
of the history of blocks and transactions between all parties, but this
so-called \emph{finality} is only achieved after some time and the local view of
the blockchain history may change until that point.

On Cardano with it's Ouroboros consensus algorithm, this means that any local
view of the mainchain may not be the longest chain and a node may switch to a
longer chain, onto another fork. This other version of the history may not
include what was previously observed and hence, any tracking state needs to be
updated to this ``new reality''. Practically, this means that an observer of the
blockchain sees a \emph{rollback} followed by rollforwards.
% TODO: mention the trade-off about waiting for finality when opening the head
% vs. issue and mark transactions as confirmed on the L2 if they were not in case
% the head opening get's rolled back and not retransmitted.

For the Head protocol, this means that chain events like $\mathtt{closeTx}$ may
be observed a second time. Hence, it is crucial, that the local state of the
Hydra protocol is kept in sync and also rolled back accordingly to be able to
observe and react to these events the right way, e.g.\ correctly contesting this
$\mathtt{closeTx}$ if need be.
% TODO: mention that contestation deadline will stay the same and hence the
% contestation period will need to be picked long enough to reduce the risk of
% not being able to contest anymore after a rollback.

The rollback handling can be specified fully orthogonal on top of the nominal
protocol behavior, if the chain provides strictly monotonically increasing
points $p$ on each chain event via a new or wrapped $\mathtt{rollforward}$
event and $\mathtt{rollback}$ event with the point to which a rollback happened:\\

\dparagraph{$\mathtt{rollforward}$.}\quad On every chain event that is paired or
wrapped in a rollforward event $(\mathtt{rollback},p)$ with point $p$, protocol
participants store their head state indexed by this point in a history
$\Omega$ of states $\Delta \gets (\hatv, \hats, \hatmU, \hatSigma, \hatmL, \hatmT, \bar{\mc S})$ and $\Omega' = (p, \Delta) \cup \Omega$. \\

\dparagraph{$\mathtt{rollback}$.}\quad On a rollback
$(\mathtt{rollback},p_{rb})$ to point $p_{rb}$, the corresponding head state
$\Delta$ need to be retrieved from $\Omega$, with the maximal point
$p \leq p_{rb}$, and all entries in $\Omega$ with $p > p_{rb}$ get removed. \\

This will essentially reset the local head state to the right point and allow
the protocol to progress through the life-cycle normally. Most stages of the
life-cycle are unproblematic if they are rolled back, as long as the protocol
logic behaves as in the nominal case.

A rollback ``past open'' is a special situation though. When a Head is open and
snapshots have been signed, but then a $\mtxCollect$ and one or more
$\mtxCommit$ transactions were rolled back, a bad actor could choose to commit a
different UTxO and open the Head with a different initial UTxO set, while the
already signed snapshots would still be (cryptographically) valid. To mitigate
this, all signatures on snapshots need to incorporate the initial UTxO set by
including $\eta_{0}$.\todo{not implemented and maybe redundant with directly open heads}


\begin{figure*}[t!]

	\def\sfact{0.8}
	\centering
	\begin{algobox}{Coordinated Hydra Head}
		\medskip
		\begin{tabular}{c}
			%%% Initializing the head
			\begin{tabular}{cc}
				\adjustbox{valign=t,scale=\sfact}{
					\begin{walgo}{0.6}
						%%% INIT
						\On{$(\hpInit)$ from client}{
							$n \gets |\hydraKeys^{setup}|$ \;
							$\hydraKeysAgg \gets \msCombVK(\hydraKeys^{setup})$ \;
							$\cardanoKeys \gets \cardanoKeys^{setup}$\;
							$\cPer \gets \cPer^{setup}$ \;
							$\PostTx{}~(\mtxInit, \nop, \hydraKeysAgg,\cardanoKeys,\cPer)$ \;
						}
						\vspace{12pt}

						\On{$(\gcChainInitial, \cid, \seed, \nop, \hydraKeysAgg, \cardanoKeys^{\#}, \cPer)$ from chain}{
						\Req{} $\hydraKeysAgg=\msCombVK(\hydraKeys^{setup})$\;
						\Req{} $\cardanoKeys^{\#}= [ \hash(k)~|~\forall k \in \cardanoKeys^{setup}]$\;
						\Req{} $\cPer=\cPer^{setup}$\;
						\Req{} $\cid = \hash(\muHead(\seed))$ \;
						}
					\end{walgo}
				}
				 &

				\adjustbox{valign=t,scale=\sfact}{
					\begin{walgo}{0.6}
						\On{$(\gcChainCommit, j, U)$ from chain}{
							$C_j \gets U $

							\If{$\forall k \in [1..n]: C_k \neq \bot$}{
								$\eta \gets \combine([C_1 \dots C_n])$ \;
								$\PostTx{}~(\mtxCCom, \eta)$ \;
							}
						}

						\vspace{12pt}

						\On{$(\gcChainCollectCom, \eta_{0})$ from chain}{
							% Implictly means that all commits will defined as we cannot miss a commit (by assumption)
							$\Uinit \gets \bigcup_{j=1}^{n} U_j$ \;
							% $\Out~(\hpSnap,(0,U_0))$ \;
							$\hatmL \gets \Uinit$ \;
							$\bar{\mc S} \gets \blue{\Sno(0, 0, [], \Uinit, \emptyset \red{, \emptyset})}$ \;
							$\hatv, \hats \gets 0$ \;
							$\hatmT \gets \emptyset$ \;
							$\tx_\omega \gets \bot$ \;
							\red{$U_\alpha \gets \emptyset$ \;}
						}

					\end{walgo}
				}
			\end{tabular}

			\\
			\multicolumn{1}{l}{\line(1,0){490}}
			\\

			%%% Open head
			\begin{tabular}{c@{}c}
				\adjustbox{valign=t,scale=\sfact}{
					\begin{walgo}{0.65}

						%%% REQ TX
						\On{$(\hpRT,\tx)$ from $\party_j$}{
							\Wait{$\hatmL \applytx \tx \neq \bot$}{
								$\hatmL \gets \hatmL\applytx\tx$ \;
								$\hatmT \gets \hatmT \cup \{\tx\}$ \;
								% issue snapshot if we are leader
								\If{$\hats = \bar{\mc S}.s \land \hpLdr(\bar{\mc S}.s + 1) = i$}{
									\Multi{} $(\hpRS,\hatv,\bar{\mc S}.s+1,\hatmT, \red{U_\alpha}, \tx_\omega)$ \;
								}
							}
						}
						\vspace{12pt}

						%%% REC DEC
						\On{$(\mathtt{reqDec},\tx)$ from $\party_j$}{
							\Wait{$\red{U_{\alpha} = \emptyset ~ \land ~}\tx_\omega  = \bot ~ \land ~ \hatmL \applytx \tx \ne \bot$}{
								$\hatmL \gets \hatmL \applytx \tx \setminus \mathsf{outputs}(\tx)$ \;
								$\tx_\omega \gets \tx$ \;
								% issue snapshot if we are leader
								\If{$\hats = \bar{\mc S}.s \land \hpLdr(\bar{\mc S}.s + 1) = i$}{
									\Multi{} $(\hpRS,\hatv,\bar{\mc S}.s+1,\hatmT, \red{U_\alpha},\tx_\omega)$ \;
								}
							}
						}
						\vspace{12pt}

						%%% DEPOSIT
						\red{
							\On{$(\mathtt{depositTx}, U)$ from chain}{
							  % FIXME: wait on chain events is a bit weird. Wait
							  % in general feels like avoiding book keeping and
							  % relying a lot on assumption of a perfect queue
							  \Wait{$\tx_\omega = \bot ~ \land ~ U_{\alpha} = \emptyset$}{
								$U_{\alpha} = U$ \;
								% issue snapshot if we are leader
								\If{$\hats = \bar{\mc S}.s \land \hpLdr(\bar{\mc S}.s + 1) = i$}{
								  \Multi{} $(\hpRS,\hatv,\bar{\mc S}.s+1,\hatmT, \red{U_\alpha}, \tx_\omega)$ \;
								}
							  }
							}
						}
						\vspace{12pt}

						%%% REQ SN
						\On{$(\hpRS,v,s,\underline{\tx}_{\mathsf{req}} \red{, U_\alpha} , \tx_\omega)$ from $\party_j$}{
							\red{\Req{$\tx_\omega = \bot ~ \lor ~ U_\alpha = \emptyset$}} \;
							\Req{$v = \hatv ~ \land ~ s = \hats + 1 ~ \land ~ \hpLdr(s) = j$} \;
							\Wait{$\hats = \bar{\mc S}.s$}{
								\blue{
									\Req{$\bar{\mc S}.U \applytx \tx_\omega \not= \bot$} \;
									$U_{\mathsf{active}} \gets \bar{\mc S}.U \applytx \tx_\omega \setminus \mathsf{outputs}(\tx_\omega)$ \;
								}
								\Req{$U_{\mathsf{active}} \applytx \underline{\tx}_{\mathsf{req}} \not= \bot$} \;
								$U \gets U_{\mathsf{active}} \applytx \underline{\tx}_{\mathsf{req}}$ \;
								$\hats \gets s$ \;
								% TODO: DRY message creation
								$\eta \gets \combine(U)$ \;
								\red{$\eta_\alpha \gets \mathsf{combine}(U_\alpha)$ \;}
								$\eta_\omega \gets \mathsf{combine}(\mathsf{outputs}(\tx_\omega))$ \;
								$\msSig_i \gets \msSign(\hydraSigningKey, (\cid || v || \hats || \eta \red{ || \eta_\alpha} || \eta_\omega))$ \;
								% TODO: use a seen snapshot to keep track of things easier
								$\hatSigma \gets \emptyset$ \;
								$\Multi{}~(\hpAS,\hats,\msSig_i)$ \;
								$\forall \tx \in \underline{\tx}_{\mathsf{req}}: \Out~(\hpSeen,\tx)$ \;
								% TODO: Should we inform users if we drop a transaction?
								% XXX: This is a bit verbose for the spec
								$\hatmL \gets U$ \;
								$X\gets\hatmT$ \;
								$\hatmT\gets\emptyset$ \;
								\For{$\tx\in X : \hatmL\applytx \tx \not=\bot$}{
									$\hatmT\gets\hatmT\cup\{\tx\}$ \;
									$\hatmL\gets\hatmL\applytx \tx$ \;
								}
							}
						}
					\end{walgo}
				} &

				\adjustbox{valign=t,scale=\sfact}{
					\begin{walgo}{0.7}
						%%% ACK SN
						\On{$(\hpAS,s,\msSig_j)$ from $\party_j$}{
							\Req{} $s \in \{\hats,\hats+1\}$ \;
							\Wait{$\hats=s$}{
								\Req{} $(j, \cdot) \notin \hatSigma$ \;
								$\hatSigma[j] \gets \sigma_{j}$ \;
								\If{$\forall k \in [1..n]: (k,\cdot) \in \hatSigma$}{
									% TODO: MS-ASig used different than in the preliminaries
									$\msCSig \gets \msComb(\hydraKeys^{setup}, \hatSigma)$ \;

									% TODO: DRY message creation
									$\eta \gets \combine(\hatmU)$ \;

									\red{$\eta_\alpha \gets \mathsf{combine}(U_\alpha)$ \;}
									$U_\omega \gets \mathsf{outputs}(\tx_\omega)$ \;
									$\eta_\omega \gets \mathsf{combine}(U_\omega)$ \;
									% NOTE: Implementation differs here and
									% below as it stores seen version in seen
									% snapshot and uses that to verify
									\Req{} $\msVfy(\hydraKeysAgg, (\cid || \blue{\hatv ||} \hats || \eta \red{|| \eta_\alpha} || \eta_\omega), \msCSig)$ \;
									% create confirmed snapshot for later reference
									\blue{$\bar{\mc S} \gets \Sno(\hatv, \hats, \hatmT, \hatmU, U_\alpha, U_\omega)$ \;}
									$\bar{\mc S}.\sigma \gets \msCSig$ \;
									%$\Out~(\hpSnap,(\bar{\mc S}.s,\bar{\mc S}.U))$ \;
									$\forall \tx \in \mT_{\mathsf{req}} : \Out (\hpConf,\tx)$ \;

									\If{${\bar S}.U_\omega \ne \bot$}{
										$\PostTx{}~(\mathtt{decrementTx}, \hatv, \hats, \eta, \red{\eta_\alpha}, \eta_\omega)$ \;
									}

									\red{\If{${\bar S}.U_\alpha \ne \bot$}{
										$\PostTx{}~(\mathtt{incrementTx}, \hatv, \hats, \eta, \red{\eta_\alpha}, \eta_\omega)$ \;
									}}

									% issue snapshot if we are leader
									\If{$\hpLdr(s+1) = i \land \hatmT \neq \emptyset$}{
										\Multi{} $(\hpRS,\hatv,\bar{\mc S}.s+1, \hatmT \red{, U_\alpha}, \tx_\omega)$ \;
									}
								}
							}
						}
						\vspace{12pt}

						%%% DECREMENT
						\On{$(\mathtt{decrementTx}, U, v)$ from chain}{
							\If{$\mathsf{outputs}(\tx_{\omega}) = U$}{
								$\tx_{\omega} \gets \bot$ \;
								$\hatv \gets v$ \;
							}
						}
						\vspace{12pt}

						%%% INCREMENT
						\red{\On{$(\mathtt{incrementTx}, U, v)$ from chain}{
							% XXX: require?
							\If{$U_\alpha = U$}{
								$\hatmL \gets \hatmL \cup U$ \;
								$U_\alpha \gets \emptyset$\;
								$\hatv \gets v$ \;
							}
						}}

					\end{walgo}
				}
			\end{tabular}
		\end{tabular}
	\end{algobox}

\end{figure*}
\clearpage
\begin{figure*}[t!]

	\def\sfact{0.8}
	\begin{tabular}{c}
		\\
		\multicolumn{1}{l}{\line(1,0){490}}
		\\
		\begin{tabular}{c c}
			\adjustbox{valign=t,scale=\sfact}{
				\begin{walgo}{0.6}
					% CLOSE from client
					\On{$(\hpClose)$ from client}{
						$\eta \gets \combine(\bar{\mc S}.U)$ \;
						\red{$\eta_\alpha \gets \combine(\bar{\mc S}.U_\alpha$) \;}
						$\eta_\omega \gets \combine(\bar{\mc S}.U_\omega)$ \;
						$\xi \gets \bar{\mc S}.\sigma$ \;
						% XXX: \hatv needed to distinguish between CloseType redeemer, explain how exactly?
						$\PostTx{}~(\mtxClose, \hatv, \bar{\mc S}.v, \bar{\mc S}.s, \eta \red{, \eta_\alpha}, \eta_\omega, \xi)$ \;
					}
				\end{walgo}
			}
			\adjustbox{valign=t,scale=\sfact}{
				\begin{walgo}{0.6}
					% CLOSE TX
					\On{$(\gcChainClose, \eta) \lor (\gcChainContest, s_{c}, \eta)$ from chain}{
						\If{$\bar{\mc S}.s > s_{c}$}{
							$\eta \gets \combine(\bar{\mc S}.U)$ \;
							\red{$\eta_\alpha \gets \combine(\bar{\mc S}.U_\alpha$) \;}
							$\eta_\omega \gets \combine({\bar{\mc S}.U_\omega})$ \;
							$\xi \gets \bar{\mc S}.\sigma$ \;
							% XXX: \hatv needed to distinguish between CloseType redeemer, explain how exactly?
							$\PostTx{}~(\mtxContest, \hatv, \bar{\mc S}.v, \bar{\mc S}.s, \eta \red{, \eta_\alpha}, \eta_\omega , \xi)$ \;
						}
					}
				\end{walgo}
			}
		\end{tabular}
	\end{tabular}
	\caption{Head-protocol machine for the \emph{coordinated head} from the perspective of party $\party_i$.}\label{fig:off-chain-prot}
\end{figure*}

%%% Local Variables:
%%% mode: latex
%%% TeX-master: "main"
%%% End:


\todo{In figure: $\combine$ on UTxO slightly different than on commits}

%%% Local Variables:
%%% mode: latex
%%% TeX-master: "main"
%%% End:
